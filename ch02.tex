
%blankpage

\chapter{Jordan Canonical Form}
\label{ch02}
\index{jordan canonical form@Jordan Canonical Form}%

\section{The Diagonalizable Case}

Although, for simplicity, most of our examples
will be over the real numbers
(and indeed over the rational numbers), we will consider that
\textit{all of our vectors and matrices
are defined over the complex numbers} $\mathbb{C}$.
It is only with this assumption that the
theory of Jordan Canonical Form (JCF) works
\index{jcf@JCF}%
completely.
See Remark~\ref{ch02.rem1} for the key reason why.

\begin{definition}
\label{ch02.def1}
If \(v \neq 0\) is a vector such that, for some
$\lambda$,
\[
A v = \lambda v,
\]
then $v$ is an \textit{eigenvector} of $A$ associated
\index{eigenvector}%
to the \textit{eigenvalue} $\lambda$.
\index{eigenvalue}%
\end{definition}

\begin{example}
\label{ch02.ex1}
Let $A$ be the matrix $A =$
Then, as you can check, if $v_1 =$ 
then $A v_1 = 3 v_1$, so $v_1$ is an eigenvector of $A$ with associated
eigenvalue $3$, and if $v_2 =$ 
then $A v_2 = -2 v_2$, so $v_2$ is
an eigenvector of $A$ with associated eigenvalue $-2$.
\end{example}

\begin{remark}
\label{ch02.rem1}
This is the customary definition of the characteristic
\index{characteristic polynomial}%
polynomial.
But note that, if $A$ is an $n$-by-$n$ matrix, then the matrix
$\lambda I - A$ is obtained from the matrix $A - \lambda I$ by multiplying
each of its $n$ rows by $-1$, and hence
In practice, it is most convenient
to work with $A -\lambda I$ in finding eigenvectors---this minimizes
arithmetic---and when we come to find
chains of generalized eigenvectors in
Section~1.2,
it is (almost) essential to use $A -\lambda I$, as using
$\lambda I - A$ would introduce lots of spurious minus signs.
\end{remark}

\clearpage

